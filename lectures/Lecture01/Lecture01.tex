% Enable hyperlinks
\setupinteraction
  [state=start,
  title={Lecture01},
  subtitle={Content Not Yet Available},
  author={Dr Gordon Wright},
  style=,
  color=,
  contrastcolor=]

% make chapter, section bookmarks visible when opening document
\placebookmarks[chapter, section, subsection, subsubsection, subsubsubsection, subsubsubsubsection][chapter, section]
\setupinteractionscreen[option={bookmark,title}]

\setuppagenumbering[location={footer,middle}]
\setupbackend[export=yes]
\setupstructure[state=start,method=auto]

% use microtypography
\definefontfeature[default][default][script=latn, protrusion=quality, expansion=quality, itlc=yes, textitalics=yes, onum=yes, pnum=yes]
\definefontfeature[default:tnum][default][tnum=yes, pnum=no]
\definefontfeature[smallcaps][script=latn, protrusion=quality, expansion=quality, smcp=yes, onum=yes, pnum=yes]
\setupalign[hz,hanging]
\setupitaliccorrection[global, always]

\setupbodyfontenvironment[default][em=italic] % use italic as em, not slanted

\definefallbackfamily[mainface][rm][CMU Serif][preset=range:greek, force=yes]
\definefontfamily[mainface][rm][Latin Modern Roman]
\definefontfamily[mainface][mm][Latin Modern Math]
\definefontfamily[mainface][ss][Latin Modern Sans]
\definefontfamily[mainface][tt][Latin Modern Typewriter][features=none]
\setupbodyfont[mainface]

\setupwhitespace[medium]

\setuphead[chapter]            [style=\tfd\setupinterlinespace,header=empty]
\setuphead[section]            [style=\tfc\setupinterlinespace]
\setuphead[subsection]         [style=\tfb\setupinterlinespace]
\setuphead[subsubsection]      [style=\bf]
\setuphead[subsubsubsection]   [style=\sc]
\setuphead[subsubsubsubsection][style=\it]

\setuphead[chapter, section, subsection, subsubsection, subsubsubsection, subsubsubsubsection][number=no]

\definedescription
  [description]
  [headstyle=bold, style=normal, location=hanging, width=broad, margin=1cm, alternative=hanging]

\setupitemize[autointro]    % prevent orphan list intro
\setupitemize[indentnext=no]

\defineitemgroup[enumerate]
\setupenumerate[each][fit][itemalign=left,distance=.5em,style={\feature[+][default:tnum]}]

\setupfloat[figure][default={here,nonumber}]
\setupfloat[table][default={here,nonumber}]

\setupxtable[frame=off]
\setupxtable[head][topframe=on,bottomframe=on]
\setupxtable[body][]
\setupxtable[foot][bottomframe=on]

\definemeasure[cslhangindent][1.5em]
\definenarrower[hangingreferences][left=\measure{cslhangindent}]
\definestartstop [cslreferences] [
		before={%
	  \starthangingreferences[left]
      \setupindenting[-\leftskip,yes,first]
      \doindentation
  	},
  	after=\stophangingreferences,
	]

\starttext
\startalignment[middle]
  {\tfd\setupinterlinespace Lecture01}
  \smallskip
  {\tfa\setupinterlinespace Content Not Yet Available}
  \smallskip
  {\tfa\setupinterlinespace Dr Gordon Wright}
  \smallskip
  {\tfa\setupinterlinespace 10, March, 2022}
  \bigskip
\stopalignment

\section[title={New Section},reference={new-section}]

\subsection[title={Bullet List (no
build)},reference={bullet-list-no-build}]

\startitemize
\item
  Point 1
\item
  Point 2
\item
  Point 3
\stopitemize

\subsection[title={Bullet List (with
build)},reference={bullet-list-with-build}]

\startitemize[packed]
\item
  List element A
\item
  List element B
\item
  List element C
\stopitemize

\subsection[title={Page with aside},reference={page-with-aside}]

Here is an important point

Additional commentary.

\subsection[title={Page with a note
comment},reference={page-with-a-note-comment}]

Here is something I say

\startblockquote
{\bf Note}

This is very noteworthy
\stopblockquote

\subsection[title={Page with a warning},reference={page-with-a-warning}]

Here is something I say

\startblockquote
{\bf Warning}

Be WARNED!!
\stopblockquote

\subsection[title={Page with an important
comment},reference={page-with-an-important-comment}]

Here is something I say

\startblockquote
{\bf Important}

This is very Important
\stopblockquote

\subsection[title={Page with a tip},reference={page-with-a-tip}]

Here is something I say

\startblockquote
{\bf Tip}

This is a useful tip
\stopblockquote

\subsection[title={Page with a caution},reference={page-with-a-caution}]

Here is something I say

\startblockquote
{\bf Caution}

This is something to be cautious about
\stopblockquote

\subsection[title={Two Columns (Text)},reference={two-columns-text}]

Left column

Right column

\subsection[title={Two Columns (Text +
Image)},reference={two-columns-text-image}]

Left column

\placefigure{LittleMonkeyLab}{\externalfigure[images/LMLLOGO.png][width=2.08333in]}

\subsection[title={Slide with different background
colour},reference={slide-with-different-background-colour}]

Shout

Question

takeaway

\useURL[url5][www.bbc.co.uk][][A link to the BBC website]\from[url5]

\subsection[title={Speaker Notes},reference={speaker-notes}]

Here is some content

Speaker notes (press \quote{s} when presenting to switch to speaker
mode).

\subsection[title={Here is a 2 panel
tabset},reference={here-is-a-2-panel-tabset}]

\subsubsection[title={Tab A},reference={tab-a}]

Content for Tab A

\subsubsection[title={Tab B},reference={tab-b}]

Content for Tab B

\subsection[title={Slide with footnote},reference={slide-with-footnote}]

Very important point\footnote{A footnote} made to the class

\section[title={Section heading 2007},reference={section-heading-2007}]

subtitle

\subsection[title={2 columns unequal 20\letterpercent{}
80\letterpercent{}},reference={columns-unequal-20-80}]

\subsubsection[title={List One},reference={list-one}]

\startitemize[packed]
\item
  Item A
\item
  Item B
\item
  Item C
\stopitemize

\subsubsection[title={List Two},reference={list-two}]

\startitemize[packed]
\item
  Item X
\item
  Item Y
\item
  Item Z
\stopitemize

\subsection[title={Level 2 centred text with break with striking
takeaway
background},reference={level-2-centred-text-with-break-with-striking-takeaway-background}]

\subsection[title={References},reference={references}]

(Andorsky, 2020; Datu et al., 2021; King, 2021; Rice et al., 2021)

\subsection[title={Speaker notes},reference={speaker-notes-1}]

Include speaker notes in another fenced code block.

Like this.

\section[title={Fragments with
entrance},reference={fragments-with-entrance}]

Fade in

Fade out

Highlight red

Highlight current red (available in green and blue)

Fade in, then out

Fade in, then semi out

Slide up while fading in

\subject[title={References},reference={references-1}]

\startcslreferences

\reference[ref-andorsky2020]{}%
Andorsky, N. (2020). {\em Decoding the why: How behavioral science is
driving the next generation of product design}.

\reference[ref-datu2021]{}%
Datu, J. A. D., McInerney, D. M., Żemojtel-Piotrowska, M., Hitokoto, H.,
& Datu, N. D. (2021). Is grittiness next to happiness? Examining the
association of triarchic model of grit dimensions with well-being
outcomes. {\em Journal of Happiness Studies}, {\em 22}(2), 981--1009.
\useURL[url6][https://doi.org/10.1007/s10902-020-00260-6]\from[url6]

\reference[ref-king2021]{}%
King, M. (2021). {\em Social chemistry: Decoding the patterns of human
connection}. Dutton.

\reference[ref-rice2021]{}%
Rice, L., Alquist, J. L., Penuliar, M., Donato, F. V., & Price, M. M.
(2021). Engaging students in a research methods writing lab online.
{\em Teaching of Psychology}, {\em 48}(1), 18--25.
\useURL[url7][https://doi.org/10.1177/0098628320959954]\from[url7]

\stopcslreferences

\stoptext
