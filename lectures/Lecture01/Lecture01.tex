% Options for packages loaded elsewhere
\PassOptionsToPackage{unicode}{hyperref}
\PassOptionsToPackage{hyphens}{url}
\PassOptionsToPackage{dvipsnames,svgnames*,x11names*}{xcolor}
%
\documentclass[12pt,
  a4paper,
  landscape]{article}
\usepackage{lmodern}
\usepackage{amsmath}
\usepackage{ifxetex,ifluatex}
\ifnum 0\ifxetex 1\fi\ifluatex 1\fi=0 % if pdftex
  \usepackage[T1]{fontenc}
  \usepackage[utf8]{inputenc}
  \usepackage{textcomp} % provide euro and other symbols
  \usepackage{amssymb}
\else % if luatex or xetex
  \usepackage{unicode-math}
  \defaultfontfeatures{Scale=MatchLowercase}
  \defaultfontfeatures[\rmfamily]{Ligatures=TeX,Scale=1}
\setmainfont{Atkinson Hyperlegible}
\setsansfont{Atkinson Hyperlegible}
\fi
% Use upquote if available, for straight quotes in verbatim environments
\IfFileExists{upquote.sty}{\usepackage{upquote}}{}
\IfFileExists{microtype.sty}{% use microtype if available
  \usepackage[]{microtype}
  \UseMicrotypeSet[protrusion]{basicmath} % disable protrusion for tt fonts
}{}
\makeatletter
\@ifundefined{KOMAClassName}{% if non-KOMA class
  \IfFileExists{parskip.sty}{%
    \usepackage{parskip}
  }{% else
    \setlength{\parindent}{0pt}
    \setlength{\parskip}{6pt plus 2pt minus 1pt}}
}{% if KOMA class
  \KOMAoptions{parskip=half}}
\makeatother
\usepackage[dvipsnames]{xcolor}
\IfFileExists{xurl.sty}{\usepackage{xurl}}{} % add URL line breaks if available
\IfFileExists{bookmark.sty}{\usepackage{bookmark}}{\usepackage{hyperref}}
\hypersetup{
  pdftitle={Lecture01},
  pdfauthor={Dr Gordon Wright},
  colorlinks=true,
  linkcolor=blue,
  filecolor=Maroon,
  citecolor=Blue,
  urlcolor=Blue,
  pdfcreator={LaTeX via pandoc}}
\urlstyle{same} % disable monospaced font for URLs
\usepackage[a4paper, margin=1in]{geometry}

\usepackage{longtable,booktabs}
\usepackage{calc} % for calculating minipage widths
% Correct order of tables after \paragraph or \subparagraph
\usepackage{etoolbox}
\makeatletter
\patchcmd\longtable{\par}{\if@noskipsec\mbox{}\fi\par}{}{}
\makeatother
% Allow footnotes in longtable head/foot
\IfFileExists{footnotehyper.sty}{\usepackage{footnotehyper}}{\usepackage{footnote}}
\makesavenoteenv{longtable}
\usepackage{graphicx}
\makeatletter
\def\maxwidth{\ifdim\Gin@nat@width>\linewidth\linewidth\else\Gin@nat@width\fi}
\def\maxheight{\ifdim\Gin@nat@height>\textheight\textheight\else\Gin@nat@height\fi}
\makeatother
% Scale images if necessary, so that they will not overflow the page
% margins by default, and it is still possible to overwrite the defaults
% using explicit options in \includegraphics[width, height, ...]{}
\setkeys{Gin}{width=\maxwidth,height=\maxheight,keepaspectratio}
% Set default figure placement to htbp
\makeatletter
\def\fps@figure{htbp}
\makeatother
\setlength{\emergencystretch}{3em} % prevent overfull lines
\providecommand{\tightlist}{%
  \setlength{\itemsep}{0pt}\setlength{\parskip}{0pt}}
\setcounter{secnumdepth}{-\maxdimen} % remove section numbering
% Make \paragraph and \subparagraph free-standing
\ifx\paragraph\undefined\else
  \let\oldparagraph\paragraph
  \renewcommand{\paragraph}[1]{\oldparagraph{#1}\mbox{}}
\fi
\ifx\subparagraph\undefined\else
  \let\oldsubparagraph\subparagraph
  \renewcommand{\subparagraph}[1]{\oldsubparagraph{#1}\mbox{}}
\fi
\pagestyle{empty}
\makeatletter
\@ifpackageloaded{tcolorbox}{}{\usepackage[many]{tcolorbox}}
\@ifpackageloaded{fontawesome5}{}{\usepackage{fontawesome5}}
\definecolor{quarto-callout-color}{HTML}{909090}
\definecolor{quarto-callout-note-color}{HTML}{0758E5}
\definecolor{quarto-callout-important-color}{HTML}{CC1914}
\definecolor{quarto-callout-warning-color}{HTML}{EB9113}
\definecolor{quarto-callout-tip-color}{HTML}{00A047}
\definecolor{quarto-callout-caution-color}{HTML}{FC5300}
\definecolor{quarto-callout-color-frame}{HTML}{acacac}
\definecolor{quarto-callout-note-color-frame}{HTML}{4582ec}
\definecolor{quarto-callout-important-color-frame}{HTML}{d9534f}
\definecolor{quarto-callout-warning-color-frame}{HTML}{f0ad4e}
\definecolor{quarto-callout-tip-color-frame}{HTML}{02b875}
\definecolor{quarto-callout-caution-color-frame}{HTML}{fd7e14}
\makeatother
\makeatletter
\makeatother
\makeatletter
\makeatother
\makeatletter
\@ifpackageloaded{caption}{}{\usepackage{caption}}
\AtBeginDocument{%
\ifdefined\contentsname
  \renewcommand*\contentsname{Table of contents}
\else
  \newcommand\contentsname{Table of contents}
\fi
\ifdefined\listfigurename
  \renewcommand*\listfigurename{List of Figures}
\else
  \newcommand\listfigurename{List of Figures}
\fi
\ifdefined\listtablename
  \renewcommand*\listtablename{List of Tables}
\else
  \newcommand\listtablename{List of Tables}
\fi
\ifdefined\figurename
  \renewcommand*\figurename{Figure}
\else
  \newcommand\figurename{Figure}
\fi
\ifdefined\tablename
  \renewcommand*\tablename{Table}
\else
  \newcommand\tablename{Table}
\fi
}
\@ifpackageloaded{float}{}{\usepackage{float}}
\floatstyle{ruled}
\@ifundefined{c@chapter}{\newfloat{codelisting}{h}{lop}}{\newfloat{codelisting}{h}{lop}[chapter]}
\floatname{codelisting}{Listing}
\newcommand*\listoflistings{\listof{codelisting}{List of Listings}}
\makeatother
\makeatletter
\@ifpackageloaded{caption}{}{\usepackage{caption}}
\@ifpackageloaded{subcaption}{}{\usepackage{subcaption}}
\makeatother
\makeatletter
\@ifpackageloaded{tcolorbox}{}{\usepackage[many]{tcolorbox}}
\makeatother
\makeatletter
\@ifundefined{shadecolor}{\definecolor{shadecolor}{rgb}{.97, .97, .97}}
\makeatother
\makeatletter
\@ifpackageloaded{sidenotes}{}{\usepackage{sidenotes}}
\@ifpackageloaded{marginnote}{}{\usepackage{marginnote}}
\makeatother
\makeatletter
\makeatother
\makeatletter
\@ifpackageloaded{fontawesome5}{}{\usepackage{fontawesome5}}
\makeatother
\ifluatex
  \usepackage{selnolig}  % disable illegal ligatures
\fi
\newlength{\cslhangindent}
\setlength{\cslhangindent}{1.5em}
\newlength{\csllabelwidth}
\setlength{\csllabelwidth}{3em}
\newenvironment{CSLReferences}[2] % #1 hanging-ident, #2 entry spacing
 {% don't indent paragraphs
  \setlength{\parindent}{0pt}
  % turn on hanging indent if param 1 is 1
  \ifodd #1 \everypar{\setlength{\hangindent}{\cslhangindent}}\ignorespaces\fi
  % set entry spacing
  \ifnum #2 > 0
  \setlength{\parskip}{#2\baselineskip}
  \fi
 }%
 {}
\usepackage{calc}
\newcommand{\CSLBlock}[1]{#1\hfill\break}
\newcommand{\CSLLeftMargin}[1]{\parbox[t]{\csllabelwidth}{#1}}
\newcommand{\CSLRightInline}[1]{\parbox[t]{\linewidth - \csllabelwidth}{#1}\break}
\newcommand{\CSLIndent}[1]{\hspace{\cslhangindent}#1}

\title{\textbf{Lecture01}}
\usepackage{etoolbox}
\makeatletter
\providecommand{\subtitle}[1]{% add subtitle to \maketitle
  \apptocmd{\@title}{\par {\large #1 \par}}{}{}
}
\makeatother
\subtitle{Content Not Yet Available}
\author{Dr Gordon Wright}
\date{10, March, 2022}

% my preamble additions =============================================================

% left justify text with ragged right edge =====================================
\usepackage[document]{ragged2e}

% page numbering ===============================================================
\usepackage{fancyhdr}
\pagestyle{fancy}
\fancyhf{}
\renewcommand{\headrulewidth}{0pt}
\fancyfoot[L]{\footnotesize\textcolor{gray}{PS52007D}}
\fancyfoot[C]{\footnotesize\textcolor{gray}{Lecture01}}
\fancyfoot[R]{\footnotesize\textcolor{gray}{\thepage}}

% change default spacing before each new paragraph (sets as text height) =======
\setlength\parskip{1em plus 0.1em minus 0.2em}

% set formatting for section and subsection headings ===========================
% adding a period after section numbering
% \usepackage{titlesec}
% \titlelabel{\thetitle.\quad}
% comment/uncomment the next line to insert a page break before each new H1
\let\oldsection\section
\renewcommand\section{\clearpage\oldsection}

% changing options for table/figure captions ===================================
% uncomment the below lines to just number figures/tables sequentially
% throughout entire document
% \renewcommand{\thetable}{\thesection.\arabic{table}}
% \renewcommand{\thefigure}{\thesection.\arabic{figure}}

\usepackage[labelfont=it,textfont=it,singlelinecheck=off,justification=raggedright,labelsep=period]{caption}

% resets table/figure counters within each section/subsection ==================
% \usepackage{chngcntr}
% \counterwithin*{table}{section}
%\counterwithin*{table}{subsection}
% \counterwithin*{figure}{section}
%\counterwithin*{figure}{subsection}

\usepackage{tcolorbox}

\definecolor{cosmiclatte}{RGB}{255, 248, 231}
\definecolor{justoffwhite}{RGB}{255, 254, 250}

%\definecolor{notes-bg}{RGB}{254, 250, 233}
%\definecolor{notes-frame}{RGB}{222, 217, 193}

%\definecolor{Questions-bg}{RGB}{236, 239, 244}
%\definecolor{Questions-frame}{RGB}{236, 239, 244}

\definecolor{notes-bg}{RGB}{244, 247, 252}
\definecolor{notes-frame}{RGB}{233, 235, 238}
\definecolor{notes-text}{RGB}{47, 54, 61}

\definecolor{slide-titles}{RGB}{47, 54, 61}

\usepackage{pagecolor}
\usepackage{afterpage}

\renewcommand{\baselinestretch}{1.5}

% resize headings =============================

\makeatletter

\renewcommand{\section}{\@startsection
{section}%                   % the name
{1}%                         % the level
{\z@}%                       % the indent / 0mm
{3\baselineskip}%            % the before skip / -3.5ex \@plus -1ex \@minus -.2ex
{3\baselineskip}%          % the after skip / 2.3ex \@plus .2ex
{\sffamily\huge\bfseries}} % the style

\renewcommand{\subsection}{\@startsection
{subsection}%                   % the name
{2}%                         % the level
{\z@}%                       % the indent / 0mm
{3\baselineskip}%            % the before skip / -3.5ex \@plus -1ex \@minus -.2ex
{.75\baselineskip}%          % the after skip / 2.3ex \@plus .2ex
{\sffamily\Large\bfseries\textcolor{slide-titles}}} % the style

\renewcommand{\subsubsection}{\@startsection
{subsubsection}%                   % the name
{3}%                         % the level
{\z@}%                       % the indent / 0mm
{3\baselineskip}%            % the before skip / -3.5ex \@plus -1ex \@minus -.2ex
{.75\baselineskip}%          % the after skip / 2.3ex \@plus .2ex
{\large\bfseries}} % the style

\makeatother

\usepackage[export]{adjustbox}

% ===================================================================================

\begin{document}


      % \maketitle
    \begin{center}
    \vspace*{5cm}
    {\Huge\bfseries Lecture01}\\
          \vfill
      {\large\bfseries PS52007D}
        \end{center}  
    % turn off page numbering for titlepage
    \thispagestyle{empty}
        \newpage
  
\ifdefined\Shaded\renewenvironment{Shaded}{\begin{tcolorbox}[interior hidden, boxrule=0pt, borderline west={3pt}{0pt}{shadecolor}, enhanced, sharp corners, breakable, frame hidden]}{\end{tcolorbox}}\fi





\newpage

\vspace*{2.5cm}

\begin{center}

\hypertarget{new-section}{%
\section{New Section}\label{new-section}}

\end{center}

\newpage

\hypertarget{bullet-list-no-build}{%
\subsection{Bullet List (no build)}\label{bullet-list-no-build}}

\begin{itemize}
\item
  Point 1
\item
  Point 2
\item
  Point 3
\end{itemize}

\newpage

\hypertarget{bullet-list-with-build}{%
\subsection{Bullet List (with build)}\label{bullet-list-with-build}}

\begin{itemize}
\tightlist
\item
  List element A
\item
  List element B
\item
  List element C
\end{itemize}

\newpage

\hypertarget{page-with-aside}{%
\subsection{Page with aside}\label{page-with-aside}}

Here is an important point

\marginnote{\begin{footnotesize}

Additional commentary.

\end{footnotesize}}

\newpage

\hypertarget{page-with-a-note-comment}{%
\subsection{Page with a note comment}\label{page-with-a-note-comment}}

Here is something I say

\begin{tcolorbox}[enhanced jigsaw, opacitybacktitle=0.6, bottomrule=.15mm, bottomtitle=1mm, breakable, colbacktitle=quarto-callout-note-color!10!white, left=2mm, coltitle=black, colframe=quarto-callout-note-color-frame, leftrule=.75mm, title=\textcolor{quarto-callout-note-color}{\faInfo}\hspace{0.5em}{Note}, titlerule=0mm, opacityback=0, toptitle=1mm, colback=white, arc=.35mm, rightrule=.15mm, toprule=.15mm]

\newpage

\hypertarget{note}{%
\subsection{Note}\label{note}}

This is very noteworthy

\end{tcolorbox}

\newpage

\hypertarget{page-with-a-warning}{%
\subsection{Page with a warning}\label{page-with-a-warning}}

Here is something I say

\begin{tcolorbox}[enhanced jigsaw, opacitybacktitle=0.6, bottomrule=.15mm, bottomtitle=1mm, breakable, colbacktitle=quarto-callout-warning-color!10!white, left=2mm, coltitle=black, colframe=quarto-callout-warning-color-frame, leftrule=.75mm, title=\textcolor{quarto-callout-warning-color}{\faExclamationTriangle}\hspace{0.5em}{Warning}, titlerule=0mm, opacityback=0, toptitle=1mm, colback=white, arc=.35mm, rightrule=.15mm, toprule=.15mm]

\newpage

\hypertarget{warning}{%
\subsection{Warning}\label{warning}}

Be WARNED!!

\end{tcolorbox}

\newpage

\hypertarget{page-with-an-important-comment}{%
\subsection{Page with an important
comment}\label{page-with-an-important-comment}}

Here is something I say

\begin{tcolorbox}[enhanced jigsaw, opacitybacktitle=0.6, bottomrule=.15mm, bottomtitle=1mm, breakable, colbacktitle=quarto-callout-important-color!10!white, left=2mm, coltitle=black, colframe=quarto-callout-important-color-frame, leftrule=.75mm, title=\textcolor{quarto-callout-important-color}{\faExclamation}\hspace{0.5em}{Important}, titlerule=0mm, opacityback=0, toptitle=1mm, colback=white, arc=.35mm, rightrule=.15mm, toprule=.15mm]

\newpage

\hypertarget{important}{%
\subsection{Important}\label{important}}

This is very Important

\end{tcolorbox}

\newpage

\hypertarget{page-with-a-tip}{%
\subsection{Page with a tip}\label{page-with-a-tip}}

Here is something I say

\begin{tcolorbox}[enhanced jigsaw, opacitybacktitle=0.6, bottomrule=.15mm, bottomtitle=1mm, breakable, colbacktitle=quarto-callout-tip-color!10!white, left=2mm, coltitle=black, colframe=quarto-callout-tip-color-frame, leftrule=.75mm, title=\textcolor{quarto-callout-tip-color}{\faLightbulb}\hspace{0.5em}{Tip}, titlerule=0mm, opacityback=0, toptitle=1mm, colback=white, arc=.35mm, rightrule=.15mm, toprule=.15mm]

\newpage

\hypertarget{tip}{%
\subsection{Tip}\label{tip}}

This is a useful tip

\end{tcolorbox}

\newpage

\hypertarget{page-with-a-caution}{%
\subsection{Page with a caution}\label{page-with-a-caution}}

Here is something I say

\begin{tcolorbox}[enhanced jigsaw, opacitybacktitle=0.6, bottomrule=.15mm, bottomtitle=1mm, breakable, colbacktitle=quarto-callout-caution-color!10!white, left=2mm, coltitle=black, colframe=quarto-callout-caution-color-frame, leftrule=.75mm, title=\textcolor{quarto-callout-caution-color}{\faFire}\hspace{0.5em}{Danger}, titlerule=0mm, opacityback=0, toptitle=1mm, colback=white, arc=.35mm, rightrule=.15mm, toprule=.15mm]

\newpage

\hypertarget{caution}{%
\subsection{Caution}\label{caution}}

This is something to be cautious about

\end{tcolorbox}

\newpage

\hypertarget{two-columns-text}{%
\subsection{Two Columns (Text)}\label{two-columns-text}}

Left column

Right column

\newpage

\hypertarget{two-columns-text-image}{%
\subsection{Two Columns (Text + Image)}\label{two-columns-text-image}}

Left column

\includegraphics[max height=0.5\paperheight]{images/LMLLOGO.png}

\newpage

\hypertarget{slide-with-different-background-colour}{%
\subsection{Slide with different background
colour}\label{slide-with-different-background-colour}}

{Shout}

{Question}

{takeaway}

\href{www.bbc.co.uk}{A link to the BBC website}

\newpage

\hypertarget{speaker-notes}{%
\subsection{Speaker Notes}\label{speaker-notes}}

Here is some content

\vfill

\begin{tcolorbox}[beforeafter skip=1cm, ignore nobreak=true, breakable, colframe=notes-frame, colback=notes-bg, coltext=notes-text, boxsep=2mm, arc=0mm, boxrule=0.5mm]

Speaker notes (press `s' when presenting to switch to speaker mode).

\end{tcolorbox}

\newpage

\hypertarget{here-is-a-2-panel-tabset}{%
\subsection{Here is a 2 panel tabset}\label{here-is-a-2-panel-tabset}}

\newpage

\hypertarget{tab-a}{%
\paragraph{Tab A}\label{tab-a}}

Content for Tab A

\newpage

\hypertarget{tab-b}{%
\paragraph{Tab B}\label{tab-b}}

Content for Tab B

\newpage

\hypertarget{slide-with-footnote}{%
\subsection{Slide with footnote}\label{slide-with-footnote}}

Very important point\footnote{A footnote} made to the class

\newpage

\vspace*{2.5cm}

\begin{center}

\hypertarget{section-heading-2007}{%
\section{Section heading 2007}\label{section-heading-2007}}

\end{center}

subtitle

\newpage

\hypertarget{columns-unequal-20-80}{%
\subsection{2 columns unequal 20\% 80\%}\label{columns-unequal-20-80}}

\newpage

\hypertarget{list-one}{%
\subsubsection{List One}\label{list-one}}

\begin{itemize}
\tightlist
\item
  Item A
\item
  Item B
\item
  Item C
\end{itemize}

\newpage

\hypertarget{list-two}{%
\subsubsection{List Two}\label{list-two}}

\begin{itemize}
\tightlist
\item
  Item X
\item
  Item Y
\item
  Item Z
\end{itemize}

\newpage

\hypertarget{level-2-centred-text-with-break-with-striking-takeaway-background}{%
\subsection{\texorpdfstring{Level 2 centred text with break with
striking takeaway
background}{Level 2 centred text with break   with striking takeaway background}}\label{level-2-centred-text-with-break-with-striking-takeaway-background}}

\newpage

\hypertarget{references}{%
\subsection{References}\label{references}}

(Andorsky, 2020; Datu et al., 2021; King, 2021; Rice et al., 2021)

\newpage

\hypertarget{speaker-notes-1}{%
\subsection{Speaker notes}\label{speaker-notes-1}}

Include speaker notes in another fenced code block.

\vfill

\begin{tcolorbox}[beforeafter skip=1cm, ignore nobreak=true, breakable, colframe=notes-frame, colback=notes-bg, coltext=notes-text, boxsep=2mm, arc=0mm, boxrule=0.5mm]

Like this.

\end{tcolorbox}

\newpage

\vspace*{2.5cm}

\begin{center}

\hypertarget{fragments-with-entrance}{%
\section{Fragments with entrance}\label{fragments-with-entrance}}

\end{center}

Fade in

\vfill

\begin{tcolorbox}[beforeafter skip=1cm, ignore nobreak=true, breakable, colframe=notes-frame, colback=notes-bg, coltext=notes-text, boxsep=2mm, arc=0mm, boxrule=0.5mm]

Fade out

\end{tcolorbox}

Highlight red

Highlight current red (available in green and blue)

Fade in, then out

Fade in, then semi out

Slide up while fading in

\newpage

\vspace*{2.5cm}

\begin{center}

\hypertarget{references-1}{%
\section*{References}\label{references-1}}
\addcontentsline{toc}{section}{References}

\end{center}

\hypertarget{refs}{}
\begin{CSLReferences}{1}{0}
\leavevmode\vadjust pre{\hypertarget{ref-andorsky2020}{}}%
Andorsky, N. (2020). \emph{Decoding the why: How behavioral science is
driving the next generation of product design}.

\leavevmode\vadjust pre{\hypertarget{ref-datu2021}{}}%
Datu, J. A. D., McInerney, D. M., Żemojtel-Piotrowska, M., Hitokoto, H.,
\& Datu, N. D. (2021). Is grittiness next to happiness? Examining the
association of triarchic model of grit dimensions with well-being
outcomes. \emph{Journal of Happiness Studies}, \emph{22}(2), 981--1009.
\url{https://doi.org/10.1007/s10902-020-00260-6}

\leavevmode\vadjust pre{\hypertarget{ref-king2021}{}}%
King, M. (2021). \emph{Social chemistry: Decoding the patterns of human
connection}. Dutton.

\leavevmode\vadjust pre{\hypertarget{ref-rice2021}{}}%
Rice, L., Alquist, J. L., Penuliar, M., Donato, F. V., \& Price, M. M.
(2021). Engaging students in a research methods writing lab online.
\emph{Teaching of Psychology}, \emph{48}(1), 18--25.
\url{https://doi.org/10.1177/0098628320959954}

\end{CSLReferences}

\end{document}
